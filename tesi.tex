% !TeX root = ./tesi.tex
% !TeX encoding = UTF-8 Unicode
% !TeX spellcheck = it_IT
% !TeX program = arara
% !TeX options = --log --verbose --language=it "%DOC%"

% arara: lualatex:      { interaction: batchmode }
% arara: frontespizio:  { interaction: batchmode, engine: lualatex }
% arara: biber
% arara: lualatex:      { interaction: batchmode }
% arara: lualatex:      { interaction: nonstopmode, synctex: yes }

\documentclass[%
  a4paper,                % formato di pagina A4
  12pt,                   % corpo del testo a 12pt
  % la dimensione 12pt automaticamente imposta \footnotesize a 10pt
  twoside,                % (oneside|twoside) documento a singola o doppia facciata,
  openright,              % (openany|openright) fa cominciare un capitolo nella successiva pagina a disposizione o sempre in una pagina destra
  % twocolumn,            % dà a LaTeX le istruzioni per comporre l'intero documento su due colonne
  titlepage,              % (titlepage|notitlepage) se dopo il titolo del documento debbaavere  inizio  una  nuova  pagina
  % fleqn,                % allinea le formule a sinistra rispetto a un margine rientrato
  % leqno,                % mette la numerazione delle formule a sinistra anziché a destra
  final                   % (draft|final) scelta tra bozza o finale, influenza il comportamento degli altri pacchetti
]{scrbook}

\usepackage{fancyvrb}       % fornisce l'ambiente VerbatimOut e modifica listati di codice
% \usepackage{minted}       % evidenzia la sintassi dei listati di codice; richiede pygments installato e shell-escape

\begin{VerbatimOut}{\jobname.xmpdata}
\Title{Titolo}
\Subject{Oggetto}
\Author{Niccolò Maltoni}
\Copyright{Questo documento è fornito sotto licenza Apache License, Version 2.0}
\CopyrightURL{https://opensource.org/licenses/Apache-2.0}
\end{VerbatimOut}

\usepackage[%
  english,italian             % definizione delle lingue da usare
]{varioref}                     % introduce il comando \vref da usarsi nello stesso modo del comune \ref per i riferimenti

\usepackage[
  rgb,                        % richiesto da pdfx
  hyperref,                   % richiesto da pdfx
  luatex,
  dvipsnames,
  table,                      % permette di colorare le tabelle
  xcdraw
]{xcolor}                       % permette di usare colori
\usepackage[a-1b]{pdfx}         % permette di generare PDF/A
\usepackage{shellesc}           % aggiunge il comando \write18 necessario su Overleaf per frontespizio

%% Font
% non è necessario \usepackage[utf8]{inputenc} perché luaLaTeX accetta solo UTF-8
\usepackage{fontspec}
\setmainfont[%
  Ligatures=TeX               % abilita legature classiche di LaTeX
]{Latin Modern Roman}           % imposta il font con grazie per il testo principale
\setsansfont[%
  Ligatures=TeX               % abilita legature classiche di LaTeX
]{Latin Modern Sans}            % imposta il font senza grazie
\setmonofont[%
  Ligatures=TeX               % abilita legature classiche di LaTeX
]{Latin Modern Mono}            % imposta il font teletype monospaziato

%% Matematica
\usepackage{amsmath}
% non bisogna assolutamente invocare il pacchetto amssymb
\usepackage[%
  math-style=ISO              % per scrivere la matematica delle scienze sperimentali bisogna seguire le norme ISO
]{unicode-math}                 % implementazione di glifi Unicode per caratteri matematici
\setmathfont[%
  Ligatures=TeX               % abilita legature classiche di LaTeX
]{Latin Modern Math}
\usepackage[%
  output-decimal-marker={,},  % le convenzioni tipografiche italiane prevedono la virgola e non il punto
  binary-units                % abilita le espressioni per bit e byte
]{siunitx}                      % permette di definire numeri con unità di misura

%% Lingue
\usepackage[%
  strict=true,                % converte tutti i warning in errori
  autostyle=true,             % adatta continuamente lo stile delle virgolette alla lingua
  english=american,           % imposta lo stile per l'inglese
  italian=guillemets          % imposta lo stile per l'italiano
]{csquotes}                     % configura le virgolette secondo gli stnadard della lingua
\usepackage{polyglossia}
\setmainlanguage[%
  babelshorthands             % attiva il carattere " come switch per virgolettature etimologiche
]{italian}                      % imposta l'italiano come lingua principale
\setotherlanguage[%
  variant=american            % imposta la variante americana dell'inglese
]{english}                      % imposta l'inglese come lingua secondaria
% non è necessario \usepackage{indentfirst} perché con lualatex il rientro del primo capoverso è preimpostato

%% Altri pacchetti
\usepackage{graphicx}           % serve per includere immagini e grafici
\graphicspath{{res/fig}}      % importa la cartella res/fig/ come cartella da cui caricare le immagini
\usepackage{subcaption}         % serve per ottenere sottofigure
\usepackage{caption}            % permette di controllare la formattazione delle didascalie
\usepackage{adjustbox}          % permette di effettuare il crop delle immagini
\usepackage{xargs}
\usepackage[
  colorinlistoftodos,
  prependcaption,
  textsize=tiny
]{todonotes}                    % permette di definire note a margine di cose da fare
\newcommandx{\unsure}[2][1=]{\todo[linecolor=red,backgroundcolor=red!25,bordercolor=red,#1]{#2}}
\newcommandx{\change}[2][1=]{\todo[linecolor=blue,backgroundcolor=blue!25,bordercolor=blue,#1]{#2}}
\newcommandx{\info}[2][1=]{\todo[linecolor=OliveGreen,backgroundcolor=OliveGreen!25,bordercolor=OliveGreen,#1]{#2}}
\newcommandx{\improvement}[2][1=]{\todo[linecolor=Plum,backgroundcolor=Plum!25,bordercolor=Plum,#1]{#2}}
% \usepackage{ctable}           % permette di migliorare la spaziatura dell'ambiente tabular standard
% \usepackage{flafter}          % impedisce alle figure di apparire prima della loro definizione nel testo
\usepackage{scrhack}            % risolve incompatibilità tra KOMA e pacchatti vari (float, listings, ...)
\usepackage{float}              % permette di forzare il posizionamento dell’oggetto nel punto in cui è situato con l’opzione H
\usepackage{afterpage}          % permette di eseguire qualcosa nella pagina successiva con \afterpage{...} (ad esempio, figure)
% \usepackage{placeins}         % permette di mettere delle barriere invalicabili per le figure con \FloatBarrier
\usepackage[%
  write,                      % (write|nowrite) genera o meno il file
  standard,                   % (standard|suftesi) specifica tipo di frontespizio
  normal,                     % (normal|sans) usa font con grazie anziché senza
  noinputenc,                 % non carica inputenc (poiché usa lualatex)
  % norules,                  % non vengono inseriti filetti nel frontespizio
  nouppercase,                % con questa opzione verrà rispettato il maiuscolo e il minuscolo
  driver=luatex               % imposta la chiamata di graphicx nel documento frn per l'uso di un driver diverso da dvips o pdftex
]{frontespizio}
\usepackage{geometry}           % permettte la modifica della gabbia del documento
\geometry{
  a4paper,                    % formato di pagina
  heightrounded,              % modifica di poco le dimensioni della gabbia per contenere un numero intero di righe
  hmargin=2.5cm,              % dimensioni margini destro-sinistro
  vmargin=2.5cm               % dimensioni margini superiore-inferiore
}
\usepackage{setspace}           % serve a fornire comandi di interlinea standard
\onehalfspacing{}             % imposta interlinea a 1,5 ed equivale a \linespread{1,5}

%% Definizioni di comandi e ambienti
%% Definisco un nuovo comando per enfatizzare il testo in inglese %%%%%%%%%%%
\newcommand{\engEmph}[1] {\emph{\foreignlanguage{english}#1}}

%% Aggiunge pagine bianche vuote %%%%%%%%%%%%%%%%%%%%%%%%%%%%%%%%%%%%%%%%%%%%
\newcommand{\clearemptydoublepage}{\newpage{\pagestyle{empty}%
%\cleardoublepage}}
\clearpage}}

%% Definisce l'environment abstract per la classe book %%%%%%%%%%%%%%%%%%%%%%
\newenvironment{abstract}%
  {\cleardoublepage%
    \thispagestyle{empty}%
    \null\vfill\begin{center}%
      \bfseries\abstractname\end{center}}%
  {\vfill\null}

\usepackage[%
  maxcitenames=2,             % massimo numero di nomi nelle citazioni
  mincitenames=2,             % minimo numero di nomi nelle citazioni
  maxbibnames=99,             % massimo numero di nomi nella blibliografia
  minbibnames=99,             % minimo numero di nomi nella blibliografia
  style=ieee,                 % imposta lo stile della blibliografia (numeric|alphabetic|authoryear|authortitle|verbose|...)
  giveninits=true,
  backend=biber               % specifica il backend per la bibliografia
]{biblatex}                     % si interfaccia con bibtex e biber per la bibliografia
\addbibresource{biblio.bib}
\usepackage[%
  % page,                     % Aggiunge una pagina con la scritta Appendices
  % toc,                      % Aggiunge un campo Appendices nell'indice
  titletoc,                   % Aggiunge la parola Appendice per ogni capitolo dell'appendice nell'indice
  title%                      % Aggiunge la parola Appendice per ogni capitolo dell'appendice
]{appendix}                     % modifica la gestione dell'appendice, e aggiunge l'ambiente appendices alternativo al comando \appendix
% \usepackage[htt]{hyphenat}    % permette la sillabazione dei blocchi di testo monospaziato
% \usepackage{enumerate}        % aggiunge un argomento opzionale che determina come comporre l’etichetta numerata delle liste

\usepackage{microtype}          % gestisce la microtipografia

% \usepackage{hyperref}         % gestisce tutte le cose ipertestuali del pdf; importato automaticamente
\hypersetup{%
  pdfpagemode={UseNone},
  hidelinks,                  % nasconde i collegamenti (non vengono quadrettati)
  hypertexnames=false,
  linktoc=all,                % inserisce i link nell'indice
  unicode=true,               % usa solo caratteri Latini nei segnalibri di Acrobat
  pdftoolbar=false,           % nasconde la toolbar di Acrobat
  pdfmenubar=false,           % nasconde il menu di Acrobat
  plainpages=false,
  breaklinks,
  pdfstartview={Fit},
  pdflang={it}
}

\usepackage[%
  english,italian,            % definizione delle lingue da usare
  nameinlink                  % inserisce i link nei riferimenti
]{cleveref}                     % permette di usare riferimenti migliori dei \ref e dei varioref

\begin{document}

  \frontmatter{}
  \pagenumbering{Roman}
  \pagestyle{empty}
  % !TeX root = ../../tesi.tex
% !TeX encoding = UTF-8 Unicode
% !TeX spellcheck = it_IT

\begin{Preambolo*}
  \usepackage{fontspec}
  \setmainfont[Ligatures=TeX]{Latin Modern Roman}
\end{Preambolo*}
\begin{frontespizio}
  \Universita{Bologna}        % aggiunge da sé “Università degli Studi di”.
  \Istituzione{%
    Alma Mater Studiorum --- Università di Bologna \\%
    Campus di Cesena%
  }
  \Divisione{Dipartimento di Informatica --- Scienza e Ingegneria}
  \Corso[Laurea triennale]{Ingegneria e Scienze Informatiche}
  \Annoaccademico{2019--2020}
  \Titolo{DS4H Image Alignment: \\
  Modulo per allineamento multimodale semi-automatico considerando deformazioni solide}
  \Sottotitolo{Tesi in Programmazione}
  % \Preambolo{\renewcommand{\frontsmallfont}[1]{\small}}       % non viene stampata la matricola
  % \Preambolo{\renewcommand{\frontsmallfont}[1]{\small Matr.}} % abbrevia la matricola
  \Candidato[694493]{Marco~Edoardo~Duma}
  \NCandidato{Presentata da}  % sostituisce la parola “Candidato”
  \Relatore{Prof.~Antonella~Carbonaro}
  \Correlatore{Prof.~Filippo~Piccinini}
  \Correlatore{Prof.~Giovanni~Martinelli}
  \Piede{%                    % sostituisce la scritta “Anno Accademico” nel piede
    III sessione di laurea \\%
    Anno Accademico 2021--2022%
  }
\end{frontespizio}

% Necessario per Overleaf: compila il TeX del frontespizio subito dopo averlo generato
\IfFileExists{\jobname-frn.pdf}{}{%
\immediate\write18{lualatex \jobname-frn}}

  % !TeX root = ../../tesi.tex
% !TeX encoding = UTF-8 Unicode
% !TeX spellcheck = it_IT

\clearemptydoublepage{}
\thispagestyle{empty}
\vspace*{20ex}
\begin{flushright}
    \begin{LARGE}
        \textbf{Parole chiave}\\
        \vspace{5ex}
    \end{LARGE}
    \begin{normalsize}
        \textbf{%
            Parola chiave 1\\%
            \medskip
            Parola chiave 2%
        }
    \end{normalsize}
\end{flushright}
\vfill

  % !TeX root = ../../tesi.tex
% !TeX encoding = UTF-8 Unicode
% !TeX spellcheck = it_IT

\clearemptydoublepage{}
\null{}\vspace{\stretch{1}}
\begin{flushright}
    \textit{Dedica}
\end{flushright}
\vspace{\stretch{2}}\null{}

  % !TeX root = ../../tesi.tex
% !TeX encoding = UTF-8 Unicode
% !TeX spellcheck = it_IT

\begin{abstract}
(EN) Most of the time, the deep analysis of a biological sample requires the acquisition of images at different time points, using different modalities and/or different stainings. These information give functional and morphological insights, but to really exploit the images acquired they must be co-registered to be then able to proceed with co-localisation analysis. Practically speaking, accordingly to the Aristotle’s principle “The whole is greater than the sum of its parts”, multi-modal image registration is the challenging task that brings to fuse together complementary signals. In the last years, several methods for image registration have been described in the literature, but unfortunately there is not one method that works for all applications. In addition, today there is no user-friendly tool for aligning images without any computer skills. In this work, besides revising all the solutions freely available for co-registering microscopy images, we describe DS4H Image Alignment, an open-source ImageJ/Fiji plugin for aligning multimodality, immunohistochemistry, and/or immunofluorescence 2D microscopy images, designed with the goal to be extremely easy-to-use.\hfill \break

\noindent (IT) Il più delle volte, l'analisi approfondita di un campione biologico richiede l'acquisizione di immagini in differenti momenti, utilizzando differenti modalità e/o colorazioni. Le informazioni acquisite forniscono dati funzionali e morfologici, ma per sfruttare realmente tutta la conoscenza, le immagini acquisite devono essere co-registrate per poter poi procedere con analisi di co-localizzazione. In pratica, secondo il principio aristotelico “Il tutto è maggiore della somma delle sue parti”, la registrazione multimodale dell'immagine è un compito impegnativo che porta a fondere insieme segnali complementari. Negli ultimi anni, sono stati descritti diversi metodi per la registrazione delle immagini, ma sfortunatamente non esiste un singolo metodo che funziona per tutte le applicazioni. Inoltre, oggi non esiste uno strumento intuitivo per l'allineamento delle immagini che non richieda importanti competenze informatiche da parte dell'utente. In questo lavoro, oltre a revisionare tutte le soluzioni disponibili gratuitamente per la co-registrazione di immagini di microscopia, descriviamo DS4H Image Alignment, un plug-in open source sviluppato per Fiji per l'allineamento di immagini 2D multimodali, di immunoistochimica e/o di immunofluorescenza, progettato con l'obiettivo di essere estremamente facile da utilizzare.
\end{abstract}

  \tableofcontents

  \mainmatter{}
  \pagenumbering{arabic}
  \pagestyle{headings}
  Minimo documento
  \appendix
  \begin{appendices}
  Appendice
\end{appendices}


  \backmatter{}
  \nocite{*}            % aggiunge tutti i riferimenti nel .bib (anche non citati)
\printbibliography[%  % produce la bibliografia
  heading=bibintoc    % inserisce il titolo nell'indice generale
]

  Ringraziamenti


\end{document}
