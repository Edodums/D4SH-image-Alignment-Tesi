\chapter*{Introduzione}
    \noindent Il progetto di Tesi \textit{``DS4H Image Alignment''} è stato sviluppato all'interno del gruppo di ricerca \textit{``Data Science for Health''} (DS4H), gruppo che raccoglie professori e ricercatori del \textit{``IRCCS Istituto Romagnolo per lo Studio dei Tumori Dino Amadori''} (IRST) di Meldola (FC) e dell'\textit{Università di Bologna} (UniBo), professionisti attivi nel proporre metodi ed applicativi informatici per risolvere problemi aperti nel campo della microscopia e collegati al mondo della medicina, biologia e fisica. In particolare, per questo progetto di Tesi, la Prof.ssa Antonella Carbonaro e il Dr. Filippo Piccinini hanno sollevato il problema e proposto le prime soluzioni teoriche, e il Professore Giovanni Martinelli e il Professore Gastone Castellani hanno consentito l’acquisizione delle immagini utilizzate partendo dalla analisi di provini istologici di pazienti.\hfill \break
    \noindent \textit{DS4H Image Alignment} è un software nato con l’intento di fornire un applicativo estremamente user-friendly che permette ai ricercatori di allineare immagini 2D di provini istologici provenienti da pazienti, al fine di valutare con semplicità cross-correlazione tra differenti marcatori tumorali. Occorre specificare che ad IRST ogni giorno i medici, biologi, fisici analizzano provini istologici applicando differenti marcatori fluorescenti per individuare differenti caratteristiche tumorali. Da questi provini vengono tipicamente acquisite immagini, dove il soggetto di base è lo stesso (\textit{i.e.} il tessuto, la biopsia), ma l’oggetto marcato è differente (ad esempio, in alcune immagini possono essere usati dei marcatori per il nucleo delle cellule, in altre dei marcatori per la membrana, in altre dei marcatori per il citoplasma). È quindi utile poter disporre di uno strumento che riesca a fornire un'immagine unica che mostri allineate tutte le precedenti per poter apprezzare le caratteristiche del tessuto sovrapponendo le informazioni riportate nelle singole immagini precedentemente acquisite.\hfill \break
    \noindent Il progetto \textit{DS4H Image Alignment} è iniziato nel 2018 con la Tesi in Web Semantico di Stefano Belli~\cite{amslaurea19123}, poi diffusa attraverso l’articolo scientifico: \textit{J. Bulgarelli, M. Tazzari, A.M. Granato, L Ridolfi, S. Maiocchi, F. De Rosa, M. Petrini, E. Pancisi, G. Gentili, B. Vergani, F. PICCININI, A. CARBONARO, B.E. Leone, G. Foschi, V. Ancarani, M. Framarini, M. Guidoboni, Dendritic cell vaccination in metastatic melanoma turns “non-T cell inflamed” into “T-cell inflamed” tumors, Frontiers in Immunology, 2019. IF2019: 4.716. DOI: 10.3389/fimmu.2019.02353}.~\cite{Bulgarelli2019-oa} \hfill \break
    \noindent La prima versione di \textit{DS4H Image Alignment} prevedeva metodiche per l’allineamento manuale di immagini 2D di provini istologici. Le immagini facevano riferimento allo stesso provino ma erano relative a differenti marcatori fluorescenti. Per questo motivo il tool è stato descritto come un applicativo per registrazione multi-modale manuale. In pratica, \textit{DS4H Image Alignment 1.0} è in grado di gestire immagini di dimensione e scala differente, ma non presenta nessuna metodica per l’allineamento automatico. L’utente doveva definire punti di interesse corrispondenti in ciascuna immagine. Invece, la versione \textit{1.0} del software ha come obiettivo principale quello di limitare al massimo il numero di interventi a carico dell'utente e rendere il processo di allineamento delle immagini quanto più automatico possibile.\hfill \break
    \noindent Nei seguenti capitoli verrà descritto il contesto del progetto e verrà illustrata nei dettagli la soluzione proposta per l’allineamento multi-modale automatico. In particolare, nel \hyperref[chap:histology]{Capitolo 1} si dettagliano alcuni cenni di immunoistochimica per meglio comprendere la generazione di immagini multimodali utilizzando differenti marcatori fluorescenti. Nel \hyperref[chap:plugins]{Capitolo 2} si introduce l’ambiente di sviluppo e distribuzione del software, meglio definito come plugin/modulo. In particolare, è stato scelto di distribuire \textit{DS4H Image Alignment} come plugin di \textit{Fiji}, una versione specifica di \textit{ImageJ} molto diffusa tra i ricercatori. Nel \hyperref[chap:descriptionoldtool]{Capitolo 3} viene descritto in dettaglio \textit{DS4H Image Alignment}, in modo da riuscire a comprendere meglio quali sono le mancanze poi superate nella successiva versione. \textit{DS4H Image Alignment 1.0} è stato poi confrontato con i tool già presenti in letteratura, descritti nel \hyperref[chap:competitors]{Capitolo 4}. Infine, nel \hyperref[chap:descriptionnewtool]{Capitolo 5} viene descritto il modulo per l’allineamento multi-modale automatico cuore di questo progetto di Tesi, poi riassunto nel rimanente Capitolo.