% !TeX root = ../../tesi.tex
% !TeX encoding = UTF-8 Unicode
% !TeX spellcheck = it_IT

\begin{abstract}
(EN) Most of the time, the deep analysis of a biological sample requires the acquisition of images at different time points, using different modalities and/or different stainings. These information give functional and morphological insights, but to really exploit the images acquired they must be co-registered to be then able to proceed with colocalization analysis. Practically speaking, accordingly to the Aristotle’s principle “The whole is greater than the sum of its parts”, multi-modal image registration is the challenging task that brings to fuse together complementary signals. In the last years, several methods for image registration have been described in the literature, but unfortunately there is not one method that works for all applications. In addition, today there is no user-friendly tool for aligning images without any computer skills. In this work, besides revising all the solutions freely available for co-registering microscopy images, we describe \textit{DS4H Image Alignment}, an open-source \textit{Fiji} plugin for aligning multimodality, immunohistochemistry, and/or immunofluorescence 2D microscopy images, designed with the goal to be extremely easy-to-use.\hfill \break

\noindent (IT) Il più delle volte, l'analisi approfondita di un campione biologico richiede l'acquisizione di immagini in differenti momenti, utilizzando differenti modalità e/o colorazioni. Le informazioni acquisite forniscono dati funzionali e morfologici, ma per sfruttare realmente tutta la conoscenza, le immagini acquisite devono essere co-registrate per poter poi procedere con analisi di co-localizzazione. In pratica, secondo il principio aristotelico “Il tutto è maggiore della somma delle sue parti”, la registrazione multimodale dell'immagine è un compito impegnativo che porta a fondere insieme segnali complementari. Negli ultimi anni, sono stati descritti diversi metodi per la registrazione delle immagini, ma sfortunatamente non esiste un singolo metodo che funziona per tutte le applicazioni. Inoltre, oggi non esiste uno strumento intuitivo per l'allineamento delle immagini che non richieda importanti competenze informatiche da parte dell'utente. In questo lavoro, oltre a revisionare tutte le soluzioni disponibili gratuitamente per la co-registrazione di immagini di microscopia, descriviamo \textit{DS4H Image Alignment}, un plug-in open source sviluppato per \textit{Fiji} per l'allineamento di immagini 2D multimodali, di immunoistochimica e/o di immunofluorescenza, progettato con l'obiettivo di essere estremamente facile da utilizzare.
\end{abstract}
