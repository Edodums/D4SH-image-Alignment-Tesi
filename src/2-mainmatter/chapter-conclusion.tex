\chapter{Conclusioni}
\label{chap:conclusion}
\noindent In questo progetto di Tesi abbiamo affrontato l’argomento “registrazione di immagini multimodali”. In particolare, ci siamo soffermati all’analisi della registrazione di immagini in due dimensioni, senza necessità di attuare correzioni elastiche per correggere eventuali deformazioni delle stesse. Il problema è un problema noto in microscopia perché spesso per analizzare un provino istologico vengono utilizzati differenti marcatori tumorali che richiedono differenti preparativi. Si acquisiscono quindi sequenze di immagini utilizzando differenti coloranti e per poter avere una visione globale e proseguire con analisi di co-localizzazione occorre poi unire le immagini acquisite in un unico riferimento.\hfill\break

\noindent Con lo scopo di rendere il processo di allineamento delle sequenze di immagini acquisite estremamente semplice, abbiamo sviluppato \textit{DS4H Image Alignment}, un tool open-source estremamente user friendly, attualmente distribuito come plugin di \textit{Fiji}, una piattaforma ben conosciuta da medici e biologi. In particolare, partendo da una versione iniziale sviluppata in precedenza dal gruppo di ricerca \textit{DS4H}, abbiamo esteso le funzionalità del tool integrando un modulo per la registrazione automatica delle immagini. Infatti, in precedenza il tool presentava una unica possibilità di registrazione che prevedeva la definizione manuale di corner point corrispondenti tra immagini seguenti e un successivo allenamento basata sulla stima ai minimi quadrati dei parametri di registrazione di una matrice affine. Oltre allo sviluppo del modulo per l’allineamento automatico, l’interfaccia grafica è stata rivoluzionata al fine di integrare una serie di opportunità per una più semplice gestione dei corner points e delle eventuali correzioni di registrazione.\hfill\break

\noindent Attualmente il tool è disponibile per due piattaforme, Windows e Mac Os X (solo per Apple Silicon), scaricabili dalla sezione Release della repository pubblica Github: \hyperref[https://github.com/UniBoDS4H/DS4H-Image-Alignment]{UniBoDS4H/DS4H-Image-Alignment}.\hfill\break

\noindent Il primo degli obiettivi futuri di questo progetto di Tesi sarà quello di rendere disponibile il tool anche per le piattaforme Linux e Mac Os X (con processori Intel). Inoltre, integreremo il plugin anche in \textit{ImageJ}, attuando un refactoring completo del codice. Al momento non è presente una divisione dei compiti in maniera netta, e il codice risulta strettamente accoppiato. Ciò significa che qualsiasi tentativo di estensione risulta oneroso in termini di tempo. Serve quindi dare una struttura chiara su cui poter lavorare per estendere con semplicità le funzionalità attualmente presenti. In particolare, l'integrazione di \textit{ImageJ2} sarà fatta attraverso l’utilizzo dei Services, con i quali si può destrutturare l'architettura presente e creare moduli, classi e metodi con una sola responsabilità. \textit{ImageJ2}, attraverso la libreria \textit{ImageJ Ops}, permette di rendere disponibile il plugin non solo su \textit{ImageJ}, introducendo così una quasi perfetta retrocompatibilità, ma anche su software come \textit{CellProfiler} e \textit{Omero}, rendendo così il software altamente interoperabile. Infine, grazie alla presenza in \textit{ImageJ/Fiji} del plugin \textit{BigWarp}, integreremo la possibilità di gestire deformazioni elastiche, eventualmente considerando le performanti \textit{``Thin plate spline''} piuttosto che le \textit{BSpline}, così da fornire un'alternativa ottimizzata rispetto alle soluzioni ora presenti in letteratura.\hfill\break

\noindent Concludendo, nonostante \textit{DS4H Image Alignment} sia da considerare ancora come un tool in fase di sviluppo, attualmente risulta essere la soluzione più semplice per biologi e medici che hanno necessita di registrare immagini 2D ottenute con seguenti acquisizioni al microscopio. Il tool è già stato utilizzato e citato in un primo articolo scientifico pubblicato in Frontiers in Immunology ed una descrizione dettagliata dello stesso verrà a breve sottomessa ad una nuova rivista. Nel frattempo, e gratuitamente disponibile per la comunità ed è quotidianamente utilizzato presso l’Istituto Romagnolo per lo Studio dei Tumori ``Dino Amadori'' IRCCS di Meldola.