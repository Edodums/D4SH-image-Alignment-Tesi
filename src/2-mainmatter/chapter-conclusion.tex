\chapter{Conclusioni}
\label{chap:conclusion}
\noindent Questo progetto di Tesi è stato per \textit{DS4H Image Alignment} solo il secondo passo di tanti passi. Al momento il plugin è disponibile all'uso per due piattaforme, Windows e Mac Os X ( solo per Apple Silicon ), scaricabili dalla sezione Release della nostra repository pubblica su Github (\hyperref[https://github.com/UniBoDS4H/DS4H-Image-Alignment]{UniBoDS4H/DS4H-Image-Alignment}). La mia intenzione come Tesista è quella di continuare a contribuire al progetto nel mio tempo libero, e come prima contribuzione voglio attuare un refactoring completo del plugin. Al momento non è presente una divisione dei compiti in maniera netta, e il codice risulta strettamente accoppiato tra le parti (\textit{tightly coupled}). Ciò significa che qualsiasi tentativo di estensione risulta oneroso in termini di tempo e serve dare una struttura chiara su cui poter lavorare anche ai prossimi Tesisti e sviluppatori. L'integrazione di \textit{ImageJ2} fa sicuramente parte dei lavori futuri, grazie all'uso dei \textbf{Services} si può destrutturare l'architettura presente e creare moduli, classi e metodi con una sola responsabilità. \textit{ImageJ2} potrebbe potrebbe permetterci attraverso la libreria \textit{ImageJ Ops} di rendere disponibile il plugin non solo su \textit{ImageJ}, introducendo così una quasi perfetta retrocompatibilità, ma anche su software come \textit{CellProfiler} e \textit{Omero}, rendendo così il nostro software interoperabile. Il plugin al momento non è disponibile per piattaforme Linux e per Mac Os X ( con processori Intel ), e proprio per questo abbiamo l'intenzione di rendere multipiattaforma \textit{DS4H Image Alignment}
Infine, si è pensa di integrare sempre nei lavori futuri l'abilità di gestire deformazioni elastiche, e a tal proposito si è pensato alle \textit{``Thin plate spline''} piuttosto che le \textit{BSpline}, così da fornire un'alternativa rispetto ai nostri competitors.