\chapter*{Ringraziamenti}
\noindent Senza le sue direttive e l'aiuto costante probabilmente non ce l'avrei fatta, ed è per questo che vorrei ringraziare uno dei miei Correlatori, il Professore Filippo Piccinini, davvero grazie di tutto, è stata lunga ma ce l'abbiamo fatta a portare a casa un ottimo risultato. Insieme a lui vorrei ringraziare la Professoressa Antonella Carbonaro, la mia Relatrice, che mi ha dato la possibilità di mostrare le mie competenze dandomi la responsabilità di continuare questo bellissimo progetto. Ringrazio anche i Professori Giovanni Martinelli e Gastone Castellani per aver reso possibile il progetto, consentendo l'utilizzo di immagini reali acquisite presso l’Istituto Romagnolo per lo Studio dei Tumori ``Dino Amadori'' IRCCS di Meldola.\hfill\break

\noindent Vorrei ringraziare inoltre i Professori con cui ho dato i miei due ultimi esami, ovvero il Professore Mirko Viroli e il Professore Alessandro Ricci che hanno riconosciuto i miei sforzi. Ringrazio i miei amici Valentina Gabellini, Giuliano Esposito, Rocco Mantovani, Giulia Belleffi, Silvia Bezzecchi, e forse è meglio che mi limiti a scrivere nomi e cognomi perché la lista finirebbe nel 2032, che mi hanno sopportato, supportato, guidato e insegnato in questo lunghissimo percorso.\hfill\break 

\noindent Ringrazio chi negli anni mi ha permesso di lavorare e accumulare esperienza come programmatore, soprattutto i miei ex-colleghi di Corsi.it, che mi hanno sostenuto e aiutato ad uscire dalla depressione, mi hanno aiutato a gestire l'ansia e mi hanno reso una persona sicuramente migliore.\hfill\break

\noindent Last but not least, la mia famiglia, il mio nucleo, mia sorella Uarda, mio padre Eduart e mia madre Mimoza, che nonostante tutte le avversità, e in generale nonostante \textbf{TUTTO}, che detta così fa molto film strappa lacrime, ma è vero, mi ha dato tutto l'aiuto di cui avevo bisogno, mi ha insegnato a reagire ai problemi e non lasciare che gli stessi ti sommergano senza più via d'uscita. Vorrei ringraziare in ultimo la mia Professoressa di Matematica del liceo, che alla fine del terzo anno mi bocciò e quando arrivai al quinto anno, nonostante non fosse più la mia docente, mi fermò e mi consiglio questo percorso universitario, vedendo in me capacità che non credevo di possedere.